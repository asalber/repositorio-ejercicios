\documentclass[12pt,a4paper]{article}
%%%%%%%%%%%%%%%%%%%%%%%%%%%%
\usepackage[spanish] {babel}
\usepackage[utf8]{inputenc}
\usepackage{amsmath}
\usepackage{macros}
\usepackage{graphicx}
\usepackage[top=3cm, bottom=3cm, left=2.54cm, right=2.54cm]{geometry}
\usepackage{fancyhdr}
\pagestyle{fancy}

\usepackage{booktabs} % Required for better horizontal rules in tables

%\usepackage{fouriernc} % Use the New Century Schoolbook font

\lhead{\textsc{\textcolor[rgb]{0.00,0.00,0.50}{Universidad San
Pablo CEU}}}
\rhead{\textsl{\textsf{\textcolor[rgb]{0.00,0.00,0.50}{Departamento
de Matemática Aplicada y Estadística}}}}
\renewcommand{\headrulewidth}{0pt}
\renewcommand{\floatpagefraction}{.8}
\renewcommand{\textfraction}{.1}

\begin{document}
\sloppy
\examen {PRIMER PARCIAL}{2º de Fisioterapia}{6 de junio 2022}{Modelo A}
\begin{enumerate}
\item Un test diagnóstico para detectar una lesión cervical tiene una
  sensibilidad del 99\% y produce un 80\% de diagnósticos acertados.
  Suponiendo que la prevalencia de la lesión es del 10\%, se pide:

  \begin{enumerate}
  \item
    Calcular la especificidad del test.
  \item
    ¿Podemos descartar la lesión cuando el test da un resultado
    negativo?
  \item
    ¿Podemos confirmar la lesión cuando el test da un resultado
    positivo? ¿Cuál sería la mínima prevalencia de la lesión para que el
    test permitiese diagnosticar la lesión?
  \end{enumerate}

\item En una farmacia se venden dos vacunas \(A\) y \(B\) contra un tipo de
  virus. Se sabe que la vacuna \(A\) produce un 5\% de efectos
  secundarios, mientras que la vacuna \(B\) produce un 2\% de efectos
  secundarios. Si se han vendido 10 unidades de la vacuna \(A\) y 100
  unidades de la vacuna \(B\), se pide:

  \begin{enumerate}
  \item
    Calcular la probabilidad de que haya menos de 2 efectos secundarios
    con la vacuna \(A\).
  \item
    Calcular la probabilidad de que haya más de 3 efectos secundarios
    con la vacuna \(B\).
  \item
    Si aplicamos ambas vacunas a una misma persona en momentos
    distintos, y suponiendo que el que una vacuna produzca efectos
    secundarios es independiente de que los produzca la otra, ¿cuál es
    la probabilidad de que esa persona tenga algún efecto secundario?
  \end{enumerate}

\item
  La longitud del hueso del fémur se distribuye normalmente tanto en
  hombres como en mujeres con una desviación típica de 4 cm. Se sabe
  además que el primer cuartil en mujeres es 42.3 cm, mientras que el
  tercer cuartil en hombres es 50.7 cm.

  \begin{enumerate}
  \item
    ¿Cuál es la diferencia entre las medias de la longitud del fémur de
    hombres y mujeres?\\
    Nota: Si no se saben calcular las medias, tomar 44 cm para la media
    de las mujeres y 47 cm para la media de los hombres en los
    siguientes apartados.
  \item
    Calcular el percentil 60 de la longitud del fémur de las mujeres.
    ¿Qué porcentaje de hombres tendrá una longitud del fémur menor que
    el percentil 60 de las mujeres?
  \item
    Si se toman un hombre y una mujer al azar, ¿cuál es la probabilidad
    de que ninguno tenga un fémur menor de 45 cm?
  \end{enumerate}

  \textbf{Solution}\\
\end{enumerate}
\end{document}
