
\begin{question}
\puntos{5}Se ha medido la tensión arterial sistólica en dos grupos de 100 personas cada uno de dos poblaciones $A$ y $B$, obteniendo los siguientes resultados:

$$
% latex table generated in R 3.6.1 by xtable 1.8-4 package
% Fri Oct 11 17:15:58 2019
\begin{array}{lrr}
 \mbox{Tensión sistólica} & \mbox{Num personas $A$} & \mbox{Num personas $B$} \\ 
  \hline
(80, 90] & 4 & 6 \\ 
  (90, 100] & 10 & 18 \\ 
  (100, 110] & 28 & 30 \\ 
  (110, 120] & 24 & 26 \\ 
  (120, 130] & 16 & 10 \\ 
  (130, 140] & 10 & 7 \\ 
  (140, 150] & 6 & 2 \\ 
  (150, 160] & 2 & 1 \\ 
   \hline
\end{array}$$

Se pide: 
\begin{enumerate}
\item ¿Cuál de las dos distribuciones de la tensión sistólica es más simétrica? ¿Cuál es más apuntada? ¿Pueden provenir estas muestras de poblaciones normales?
\item ¿En cuál de los dos grupos es más apuntada la distribución de la tensión sistólica?
\item ¿En cuál de los dos grupos es más representativa la media de la tensión sistólica?
\item ¿Qué tensión sistólica es relativamente más alta, 132 en el grupo de la población $A$, o 130 en el grupo de la población $B$?
\item Si a las 100 peronas de la población $A$ se les mide la tensión sistólica con otro tensiómetro, y la tensión obtenida ($Y$) está relacionada con la del primer tensiómetro ($X$) mediante la ecuación $y=0.98x-1.4$, ¿en cuál de las dos tensiones $X$ o $Y$ es más representativa la media?   
\end{enumerate}

Usar las siguientes sumas para los cálculos.\\
Población $A$: $\sum x_i=11520$ mmHg, $\sum x_i^2=1351700$ mmHg$^2$, $\sum (x_i-\bar x)^3=155241.6$ mmHg$^3$ y $\sum (x_i-\bar x)^4=16729903.52$ mmHg$^4$.\\
Población $B$: $\sum x_i=11000$ mmHg, $\sum x_i^2=1230300$ mmHg$^2$, $\sum (x_i-\bar x)^3=165000$ mmHg$^3$ y $\sum (x_i-\bar x)^4=13632500$ mmHg$^4$.
\end{question}


%% SOLUTIONS
\begin{solution}
\begin{enumerate}
\item Sean $x$ e $y$ las tensiones sistólicas de las poblaciones $A$ y $B$ respectivamente.\\
$g_{1x}=0.4024$ y $g_{1y}=0.5705$, de manera que la distribución del grupo de la población $A$ es más asimétrica ya que el coeficiente de $g_1$ está más lejos de 0.
\item $g_{2x}=-0.2346$ y $g_{2y}=0.3081$, de manera que la distribución del grupo de la población $A$ es más apuntada que la de la población $B$ ya que $g_{2x}>g_{2y}$.
\item $\bar x=115.2$ mmHg, $s_x^2=245.96$ mmHg$^2$, $s_x=15.6831$ mmHg and $cv_x=0.1361$.\\
$\bar y=110$ mmHg, $s_y^2=203$ mmHg$^2$, $s_y=14.2478$ mmHg y $cv_y=0.1295$.\\
La media es más representativa en el grupo de la población $A$ ya que el coeficiente de variación es menor.
\item $g_{1x}=0.4024$ y $g_{1y}=0.5705$, de manera que la distribución de edades de los pacientes menores de 65 es menos simétrica.
\item Las puntuaciones típicas son $z_x(132)=1.0712$ y $z_y(130)=1.4037$, de manera que 132 mmHg es relativamente menor en el grupo de la población $A$ ya que su puntación típica es menor.
\end{enumerate}
\end{solution}

%% META-INFORMATION
%% \extype{num}
%% \exsolution{1}
%% \exname{des-gen-6}
%% \extol{0.001}
%% \degree{Medicina}
 
